 %% ============================================================================
  % !Mode:: "TeX:US:UTF-8"
  % !TEX program = XeLaTeX
  % !BIB program = biblatex
  % -----------------------------------------------------------------------------
\NeedsTeXFormat{LaTeX2e}
\documentclass[presentation,hideothersubsections,final,10pt,american]{beamer}
% LaTeX course based on the
% /info/german/tex-kurs by Rainer Rupprecht (1988) and on /info/lshort/ by Schmidt et al., 2003.
% This material may be used under Gnu Copyleft

% everyone:
\usepackage{Eigene}

\mode<article>{\usepackage{fullpage}}
\mode<presentation>{\usetheme{Goettingen}\usecolortheme{Eigen}}
\mode<handout>{\setbeamercolor{background canvas}{bg=black!5}}
\beamertemplatenavigationsymbolsempty

\frenchspacing

\title[The \alert{\LaTeX} document preparation system]{The \alert{\LaTeX} document preparation system}
\author[E. Buxbaum]{Engelbert Buxbaum, {\tiny Dr. rer. nat., Dipl. Biol.}}
\date{July 2008}

\begin{document}
\setbeamertemplate{footline}
  {%
    \insertframenumber
  }

\frame{\titlepage}

\section{Introduction}

\begin{frame}
  \frametitle{Name of the game}
  \begin{description}
    \item[\TeX]{typesetting system developed by
        \textsc{Donald E. Knuth} (Stanford University) to
        create beautiful documents, especially those
        containing maths. \TeX\ is free software with
        copyright vested in the American Mathematical
        Society.}
    \item[\LaTeX]{\TeX -macroprocessor written by \textsc{Leslie
        Lamport}, which implements a markup-language (similar: HTML, XML). Users can
        concentrate on the structure of their document rather than on
        formatting.}
  \end{description}
\end{frame}

\begin{frame}
  \frametitle{Typographic design}
  \begin{itemize}
    \item{craft that needs to be learned}
    \item{not about æsthetics but about function: Books are
        read, not displayed in a museum}
  \end{itemize}
\end{frame}

\subsection{Advantages and disadvantages}

\begin{frame}
  \frametitle{Advantages}
  \begin{itemize}
     \item{Several professional styles are available that
         make documents look ``like printed''. Changing
         style requires changing one single line in the
         document, consistency is ensured.}
     \item{High quality math typesetting}
     \item{Only a few commands to define the structure of
         text, no knowledge of typography or book design
         required.}
     \item{Complex scientific documents can be created
         automatically:
         \begin{itemize}
           \item{bibliography}
           \item{index}
           \item{crossreferences}
           \item{table of contents, lists of figures,
               tables etc.}
           \item{...}
         \end{itemize} }
     \item{Operating-system independent}
     \item{Long-term storage of documents: ASCII rather
         than binary}
     \item{Free software with source code available: Errors
         are corrected rapidly}
  \end{itemize}
\end{frame}

\begin{frame}
  \frametitle{Disadvantages}
  \begin{itemize}
    \item{Learning curve}
    \item{Major changes in layout may require rewriting the
        style file (blessing in disguise)}
    \item{One gets an eye for all the bad documents out
        there}
  \end{itemize}
\end{frame}

\subsection{Resources}

\begin{frame}
  \frametitle{Resources}
  \begin{description}
    \item[Software]{\href{www.ctan.org}{\texttt{www.ctan.org}}, automatically selects nearest server }
    \item[Help]{Usenet: comp.text.tex (also in other languages like de.comp.text.tex), \href{https://tex.stackexchange.com/}{\texttt{tex.stackexchange.com}}}
    \item[Introduction]{\href{https://ctan.org/pkg/lshort}{\texttt{l2short}} available in various languages}
    \item[Literature]{\textit{The \LaTeX -companion} (\Name{Mittelbach} \Foreign{et al.}, 2004), \textit{The \LaTeX\ Graphics-companion} (\Name{Goosens} \Foreign{et al.}, 2022)}
    \item[Symbols]{\href{https://ctan.org/pkg/comprehensive}{\texttt{Comprehensive symbol list}} }
    \item[Online editors]{\href{https://www.overleaf.com}{\texttt{Overleaf}}, \href{https://www.latex4technics.com/}{\texttt{latex4technics}} or \href{https://artofproblemsolving.com/texer}{\texttt{TeXer}} (some require registration, optionally use fake-email for example \href{https://www.trash-mail.com/}{\texttt{trash-mail.com}})}
    \item[\TeX-aware Editors]{\href{https://www.winedt.com/}{\texttt{WinEdt}} (shareware), \href{https://www.tug.org/texworks/}{\texttt{TeXworks}} (open source), \href{https://github.com/ScientificWord/sciword}{\texttt{Scientific Word}} (WYSIWYG) }
  \end{description}
\end{frame}

\section{Structure of a \LaTeX\ document}

\begin{frame}
  \frametitle{Boxes and glue}
  \begin{itemize}
    \item{Box (letter) + Glue (space) + Box + ... = larger
        Box (word)}
    \item{Words are treated as boxes to form lines, lines
        paragraphs and paragraphs pages}
    \item{Glue can shrink and expand within limits}
  \end{itemize}
\end{frame}

\begin{frame}[fragile]
  \frametitle{Structure of a \LaTeX -document}
  {\color{DarkGreen} \begin{verbatim}
    % -----------------------------------------------------------------------------
    % !Mode:: "TeX:US:UTF-8"          % for text editor: mode, 
    %                                 % language and character set
    % !TEX program = XeLaTeX          % TeX system used
    % !BIB program = biblatex         % system for references
    % -----------------------------------------------------------------------------
    \NeedsTeXFormat{LaTeX2e}          % for TeX-system
    \documentclass[options]{style}

    \usepackage{Eigene}

    \author{}
    \title{}
    \date{}

    \begin{document}
    \maketitle

    \chapter{}
    ...
    \end{document}
  \end{verbatim} 
  }
\end{frame}

\subsection{classes}

\begin{frame}
  \frametitle{Document classes}
  \begin{tabular}{llp{0.5\textwidth}}
     \toprule
     \LaTeX & Koma-Script & purpose \\
     \midrule
     article & scrartcl & journal papers, short reports \\
     report  & scrreprt & longer text with several chapters, e.g. thesis \\
     book    & scrbook  & books \\
     letter  & scrlttr2 & letters \\
     beamer  &          & slide presentations \\
     sciposter &        & conference posters \\
     \bottomrule
  \end{tabular}

  Apart from Koma-Script another alternative style package is \alert{memoir}. Also publisher-specific styles (\Foreign{e.g.}, Springer, Elsevier, Teubner \Foreign{etc}.)
\end{frame}

\begin{frame}
  \frametitle{Class options}
  \begin{description}
    \item[Font size]{10pt | 11pt | 12pt...}
    \item[Paper size]{a4paper | legalpaper...}
    \item[equations]{fleqn, leqno }
    \item[title]{titlepage |  notitlepage}
    \item[columns]{onecolumn | twocolumn}
    \item[printing]{oneside | twoside}
  \end{description}
\end{frame}

\section{Basic commands}

\subsection{Special characters}

\begin{frame}[fragile]
  \frametitle{Input characters}
  Some characters have special meaning in \TeX , if you need
  them they have to be entered as \TeX -commands:

  \vspace{5mm}
  \begin{tabular}{lll}
    \(\backslash\)   & start command        & {\color{DarkGreen}\verb+\(\backslash\)+}    \\
                     &                      & note: {\color{DarkGreen}\verb+\\+} = new line \\
    \$               & toggle math modus    & {\color{DarkGreen}\verb+\$+}              \\
    \&               & tabulator            & {\color{DarkGreen}\verb+\&+}              \\
    \%               & rest of line comment & {\color{DarkGreen}\verb+\%+}              \\
    \#               &                      & {\color{DarkGreen}\verb+\#+}              \\
    \textasciitilde  &                      & {\color{DarkGreen}\verb+\textasciitilde+} \\
    \textbar         & vert. lines in table & {\color{DarkGreen}\verb+\textbar+}        \\
    \_               & start subscript      & {\color{DarkGreen}\verb+\_+}              \\
    \textasciicircum & start superscript    & {\color{DarkGreen}\verb+\textasciicircum+}\\
    \{ \}            & command delimiter    & {\color{DarkGreen}\verb+\{ \}+}           \\
    \([\ ]\)         & optional arguments   & {\color{DarkGreen}\verb+\([ ]\)+}         \\
    `` ''            & quotation marks      & {\color{DarkGreen}\verb+`` ´´+}           \\
    $> <$            & tabbing              & {\color{DarkGreen}\verb+\(> <\)+}         \\
  \end{tabular}
\end{frame}

\subsection{Hypens \& Cie}

\begin{frame}[fragile]
  \frametitle{Hypen, minus ...}
  A horizontal line can mean a lot of things, depending on
  length and thickness:

  \vspace{5mm}
  \begin{tabular}{lll}
    O-legs           & {\color{DarkGreen}\verb+O-legs+}           \\
    10--18 o'clock   & {\color{DarkGreen}\verb+10--18 o'clock+}   \\
    ja -- oder nein? & {\color{DarkGreen}\verb+ja -- oder nein?+} \\
    yes---or no?     & {\color{DarkGreen}\verb+yes---or no?+}     \\
    0, 1 and --1     & {\color{DarkGreen}\verb+0, 1 and --1+}     \\
  \end{tabular}
\end{frame}

\section{Structuring text}

\subsection{Sectioning}

\begin{frame}[fragile]
  \frametitle{Sectioning commands}
  \begin{itemize}
    \item{\alert{\(\backslash\)part\{\}}}
    \item{\alert{\(\backslash\)chapter\{\}}}
    \item{{\color{DarkGreen}\(\backslash\)section\{\}}}
    \item{{\color{DarkGreen}\(\backslash\)subsection\{\}}}
    \item{{\color{DarkGreen}\(\backslash\)subsubsection\{\}}}
    \item{{\color{DarkGreen}\(\backslash\)paragraph\{\}}}
  \end{itemize}

  Parts available only in books, chapters in books and reports.
\end{frame}

\begin{frame}
  \frametitle{Breaking down large documents}
  Use separate files for, say, each chapter. One main file with
  limbo and commands to include the others:
  \begin{description}
    \item[{\color{DarkGreen}\(\backslash\)input\{\}}]{reads file ``as is''}
    \item[{\color{DarkGreen}\(\backslash\)include\{\}}]{equiv. to
        {\color{DarkGreen}\(\backslash\)clearpage \(\backslash\)input\{\}
        \(\backslash\)clearpage}}
    \item[{\color{DarkGreen}\(\backslash\)includeonly\{\}}]{used in limbo to
        limit files {\color{DarkGreen}\(\backslash\)included} }
  \end{description}
\end{frame}

\begin{frame}
  \frametitle{What is where?}
    \begin{itemize}
        \item{{\color{DarkGreen}\(\backslash\)tableofcontents}}
        \item{{\color{DarkGreen}\(\backslash\)listoffigures}}
        \item{{\color{DarkGreen}\(\backslash\)listoftables}}
        \item{}
        \item{{\color{DarkGreen}\(\backslash\)addbibresource\{bibfile.bib\}}}
        \item{{\color{DarkGreen}\(\backslash\)begin\{refsection\}}}
        \item{{\color{DarkGreen}\(\backslash\)printbibliography[heading=subbibliography]}}
        \item{{\color{DarkGreen}\(\backslash\)end\{refsection\}}}
        \item{}
        \item{{\color{DarkGreen}\(\backslash\)printindex}}
    \end{itemize}
\end{frame}

\subsection{Fonts}

\begin{frame}
  \frametitle{Emphasizing}
    \begin{description}
       \item[{\color{DarkGreen}\(\backslash\)textit\{\}}]{italics, used for
           foreign words, species names etc: \textit{Staph.
           aureus}}
       \item[{\color{DarkGreen}\(\backslash\)textsl\{\}}]{\textsl{slanted}}
       \item[{\color{DarkGreen}\(\backslash\)emph\{\}}]{used for emphasizing:
           this is \emph{not} the case}
       \item[{\color{DarkGreen}\(\backslash\)textsc\{\}}]{small caps, used for
           persons: \textsc{Neil Armstrong} was the first man on the moon.}
       \item[{\color{DarkGreen}\(\backslash\)textbf\{\}}]{bold face: used to
           make something really \textbf{stick out}.}
       \item[{\color{DarkGreen}\(\backslash\)textsf\{\}}]{sans serif, often used
           as base font on slides. Also used for chemical
           equations.}
       \item[{\color{DarkGreen}\(\backslash\)texttt\{\}}]{typewriter, used for
           computer code or URLs: \texttt{https://github.com/ebuxbaum/LaTeX-Course}}
    \end{description}
    \alert{Note:} Slides use sans-serif font: slanted instead of italics!
\end{frame}

\begin{frame}[fragile]
  \frametitle{Font sizes}
    {\begin{tabbing} \texttt{xfootnotesizexx}\= und dann der Text
        \kill
        {\color{DarkGreen}\verb|\tiny|}         \> \tiny         microscopic font           \\
        {\color{DarkGreen}\verb|\scriptsize|}   \> \scriptsize   very tiny font (subscripts)\\
        {\color{DarkGreen}\verb|\footnotesize|} \> \footnotesize tiny font (footnotes)      \\
        {\color{DarkGreen}\verb|\small|}        \> \small        small font                 \\
        {\color{DarkGreen}\verb|\normalsize|}   \> \normalsize   normal font                \\
        {\color{DarkGreen}\verb|\large|}        \> \large        large font                 \\
        {\color{DarkGreen}\verb|\Large|}        \> \Large        larger font                \\
        {\color{DarkGreen}\verb|\LARGE|}        \> \LARGE        very large font            \\[3pt]
        {\color{DarkGreen}\verb|\huge|}         \> \huge         huge font                  \\[3pt]
        {\color{DarkGreen}\verb|\Huge|}         \> \Huge         very huge font
    \end{tabbing}}
    Note: \alert{not} a command: {\color{DarkGreen}\verb+{\small foo bar}+}
\end{frame}

\subsection{Lists}

\begin{frame}[fragile]
  \frametitle{Simple lists}
    Please believe me:
    \begin{itemize}
        \item{Few swallows can turn winter into summer.}
        \item{Inside it's colder than in the night.
            \begin{itemize}
                \item{In the morning it pulls.}
                \item{At noon he pushes.}
                \item{In the evening she goes.}
                \end{itemize} }
        \item{Every nonsense must find an end.}
    \end{itemize}
    {\color{DarkGreen}\small \begin{verbatim}
Please believe me:
\begin{itemize}
    \item{Few swallows can turn winter into summer.}
    \item{Inside it's colder than in the night.
        \begin{itemize}
            \item{In the morning it pulls.}
            \item{At noon he pushes.}
            \item{In the evening she goes.}
        \end{itemize} }
    \item{Every nonsense must find an end.}
\end{itemize}
    \end{verbatim} 
    }
\end{frame}

\begin{frame}[fragile]
  \frametitle{Descriptive lists}
    Three animals you should know about are:
    \begin{description}
        \item[gnat:]{A small animal, found in the North
            Woods, that causes no end of trouble.}
        \item[gnu:]{A large animal, found in crossword
            puzzles, that causes no end of trouble.}
        \item[armadillo:]{A medium-sized animal, named
            after a medium-sized Texas city which causes no
            end of trouble.}
    \end{description}
    {\color{DarkGreen}\small \begin{verbatim}
\begin{description}
    \item[gnat:]{A small animal, found in the North
        Woods, that causes no end of trouble.}
    \item[gnu:]{A large animal, found in crossword
        puzzles, that causes no end of trouble.}
    \item[armadillo:]{A medium-sized animal, named
        after a medium-sized Texas city which causes
        no end of trouble.}
\end{description}
\end{verbatim}
 }
\end{frame}

\begin{frame}[fragile]
  \frametitle{Enumerated lists}
    These are the main points:
    \begin{enumerate}
        \item{first item}
        \item{second item}
        \item{third item
           \begin{enumerate}
             \item{first sub-item}
             \item{second sub-item}
           \end{enumerate} }
    \end{enumerate}
    {\color{DarkGreen}\small \begin{verbatim}
These are the main points:
\begin{enumerate}
    \item{first item}
    \item{second item}
    \item{third item
       \begin{enumerate}
         \item{first sub-item}
         \item{second sub-item}
       \end{enumerate} }
\end{enumerate}
\end{verbatim} 
}
\end{frame}

\subsection{Tables and stuff}

\begin{frame}[fragile]
  \frametitle{Tabbing}
    \begin{tabbing}
        If \= it's raining           \\
            \> then \= put on boots, \\
            \>      \> take hat;     \\
            \> else \> smile.        \\
        Leave house.
    \end{tabbing}
    {\color{DarkGreen}\small \begin{verbatim}
\begin{tabbing}
    If \= it's raining           \\
        \> then \= put on boots, \\
        \>      \> take hat;     \\
        \> else \> smile.        \\
    Leave house.
\end{tabbing}
    \end{verbatim}
     }
\end{frame}

\begin{frame}[fragile]
  \frametitle{Tabular}
    \begin{tabular}{|r||c|p{2.5in}|}
        \toprule
        \multicolumn{3}{|c|}{\sc GG\&A Hoofed Stock} \\
        \midrule
        \midrule
        \multicolumn{1}{|c||}{\bf Year} & \bf Price   & \multicolumn{1}{c|}{\bf Comments} \\
        \midrule
        1971 & 97--245   & Bad year for farmers in the west.     \\ \midrule
          72 & 245--245  & Light trading due to a heavy winter.  \\ \midrule
          73 & 245--2001 & No gnus was very good gnus this year. \\ \bottomrule
    \end{tabular}
    {\color{DarkGreen}\scriptsize \begin{verbatim}
\begin{tabular}{|r||c|p{2.5in}|}
    \toprule
    \multicolumn{3}{|c|}{\sc GG\&A Hoofed Stock} \\
    \midrule
    \midrule
    \multicolumn{1}{|c||}{\bf Year} & \bf Price   & \multicolumn{1}{c|}{\bf Comments} \\
    \midrule
    1971 & 97--245   & Bad year for farmers in the west.     \\
    \midrule
      72 & 245--245  & Light trading due to a heavy winter.  \\
    \midrule
      73 & 245--2001 & No gnus was very good gnus this year. \\
    \bottomrule
\end{tabular}
    \end{verbatim} 
    }
    Note: this example shows what can be done, not what should be!
\end{frame}

\begin{frame}[fragile]
  \frametitle{Table (a float: not used on slides)}
    {\color{DarkGreen}\small \begin{verbatim}
\begin{table}
    \caption{The fastest man in the world:
        Some of his possibilities}
    \label{tab:fast}
    \centering
    \begin{tabular}{|l|c|r|}
        \toprule
        Disciplin & distance (m) & time (min) \\
        \midrule
        Running   & 100          &  5         \\
        Swimming  & 50           & 30         \\
        Cycling   & 1000         & 20         \\
        \bottomrule
    \end{tabular}
\end{table}
    \end{verbatim}
     }
    In the text the table can be referenced with: see table
    {\color{DarkGreen}\verb+\ref{tab:fast}+} on page {\color{DarkGreen}\verb+\pageref{tab:fast}+}.
\end{frame}

\subsection{Graphics}

\begin{frame}[fragile]
  \frametitle{Graphics}
  \includegraphics[height=0.3\textheight]{Graphics/BandedCoralShrimp}

  {\color{DarkGreen}\footnotesize \begin{verbatim}
\includegraphics[height=0.3\textheight]{Graphics/BandedCoralShrimp}
  \end{verbatim} 
  }
  \begin{itemize}
     \item{Requires \verb+\usepackage{graphicx}+}
     \item{Several file formats possible depending on
         dvi-driver. For pdfLaTeX pdf, png, jpg. }
     \item{Convert other formats \textit{e.g.}, with IrfanView or
         Gimp.}
     \item{other optional arguments like width, angle,
         size}
  \end{itemize}
\end{frame}

\begin{frame}[fragile]
  \frametitle{Figures }
    {\color{DarkGreen}\scriptsize \begin{verbatim}
\begin{figure}
    \caption{A ``pregnant´´banded coral shrimp  (aka barber pole shrimp, 
       \Species{Stenopus hispidus} (Olivier 1811), Stenopodidae) }
    \label{fig:Shrimp}
    \centering
        \includegraphics[height=0.3\textheight]{Graphics/BandedCoralShrimp}
\end{figure}
    \end{verbatim}
     }
 Like table, figure is a floating environment that has no
 meaning in slides. Cross-referencing works as with tables.
\end{frame}

\section{Maths}

\subsection{Math environments}

\begin{frame}[fragile]
  \frametitle{Text maths}
    if $a$ and $b$ are legs of a
    right-angled triangle and $c$ the hypotenuse, then
    $c^2=a^2+b^2$ (Theorem of Pythagoras).

    {\color{DarkGreen}\small \begin{verbatim}
if $a$ and $b$ are legs of a right-angled triangle
and $c$ the hypotenuse, then $c^2=a^2+b^2$ (Theorem
of Pythagoras).
    \end{verbatim} 
    }
\end{frame}

\begin{frame}[fragile]
  \frametitle{Display maths}
    if $a$ and $b$ are legs of a
    right-angled triangle and $c$ the hypotenuse, then
    \begin{equation}
       c^2=a^2+b^2
    \end{equation}
    (Theorem of Pythagoras).

    {\color{DarkGreen}\small \begin{verbatim}
if $a$ and $b$ are legs of a right-angled triangle
and $c$ the hypotenuse, then
\begin{equation}
    c^2=a^2+b^2
\end{equation}
(Theorem of Pythagoras).
    \end{verbatim} 
    }
\end{frame}

\subsection{Basic math commands}

\begin{frame}[fragile]
  \frametitle{Formulas}
    \begin{displaymath}
        x^5 \qquad \qquad x_1 \qquad \qquad \sqrt{x^2+\sqrt[3]{y}}
    \end{displaymath}
    {\color{DarkGreen}\begin{verbatim}
x^5     x_1       \sqrt{x^2+\sqrt[3]{y}}
    \end{verbatim}
    }

    \begin{displaymath}
        \frac{1}{\frac{x^2+y^2+z^2}{x+y}} \qquad \qquad {n\choose {n-k}}
    \end{displaymath}
    {\color{DarkGreen}\begin{verbatim}
\frac{1}{\frac{x^2+y^2+z^2}{x+y}}  {n\choose {n-k}}
    \end{verbatim}
    }

    \begin{displaymath}
        \int \limits_{-\infty}^{\infty}x^3 \qquad \qquad \sum_{i=1}^{n}a_i
    \end{displaymath}
    {\color{DarkGreen}\begin{verbatim}
\int \limits_{-\infty}^{\infty}x^3  \sum_{i=1}^{n}a_i
    \end{verbatim}
    }
\end{frame}

\begin{frame}[fragile]
  \frametitle{Aligning equations}
    \begin{align}
        f(x)                & =  \cos x \\
        f'(x)               & =  -\sin x \\
        \int_{0}^{x} f(y)dy & =  \sin x
    \end{align}
    {\color{DarkGreen}\small \begin{verbatim}
\begin{align}
    f(x)                & =  \cos x \\
    f'(x)               & =  -\sin x \\
    \int_{0}^{x} f(y)dy & =  \sin x
\end{align}
    \end{verbatim} 
    }
\end{frame}

\section{Changing the layout}

\subsection{Counters and parameters}

\begin{frame}[fragile]
  \frametitle{Internal Counter + parameters}
    Change parameters:
    {\color{DarkGreen}\begin{verbatim}
\setlength{\parindent}{0pt}
\setlength{\parskip}{5pt plus 2pt minus 1pt}
\addtolength{\textwidth}{60pt}
    \end{verbatim}
    }

    Set counters:
    {\color{DarkGreen}\begin{verbatim}
\setcounter{page}{0}
\addtocounter{page}{10}
    \end{verbatim}
    }

    Output counter content: This is page \thepage.\\
    {\color{DarkGreen}\verb+This is page \thepage.+}
\end{frame}

\subsection{Distances}

\begin{frame}[fragile]
  \frametitle{Horizontal distance}
    Here we have \hspace{2cm} \SI{2}{cm} distance.

    {\color{DarkGreen}\begin{verbatim}
Here we have \hspace{2cm} \SI{2}{cm} distance.
    \end{verbatim}
    }

    left\hfill right
    {\color{DarkGreen}\begin{verbatim}
left\hfill right
    \end{verbatim}
    }
    \begin{tabbing}
        \texttt{xenspace}\qquad \= \kill
        {\color{DarkGreen}\verb|\,|}       \> very small distance \\
        {\color{DarkGreen}\verb|\enspace|} \> as wide as a number \\
        {\color{DarkGreen}\verb|\quad|} \> as wide as a letter is heigh \\
        {\color{DarkGreen}\verb|\qquad|} \> twice as wide as \verb|\quad| \\
        {\color{DarkGreen}\verb|\hfill|} \> a distance that can expand from 0\\
                     \> to $\infty$ \\
    \end{tabbing}
\end{frame}

\begin{frame}[fragile]
  \frametitle{Vertical distance}
    Here

    \vspace{2cm}

    is \SI{2}{cm} distance.
    {\color{DarkGreen}\begin{verbatim}
Here

\vspace{2cm}

is \SI{2}{cm} distance.
    \end{verbatim}
    }

    \begin{tabbing}
        \texttt{xsmallskip}\qquad \= \kill
        {\color{DarkGreen}\verb|\smallskip|} \> about $1/4$ line \\
        {\color{DarkGreen}\verb|\medskip|} \> about $1/2$ line \\
        {\color{DarkGreen}\verb|\bigskip|} \> about 1 line \\
        {\color{DarkGreen}\verb|\vfill|} \> distance that can expand from 0\\
               \> to $\infty$ \\
    \end{tabbing}
\end{frame}

\subsection{Text position}

\begin{frame}[fragile]
  \frametitle{Centering text}
    \begin{center}
        In\\
        the\\
        middle I don't\\
        feel\\
        so marginalized\\
    \end{center}
    {\color{DarkGreen}\begin{verbatim}
\begin{center}
    In\\
    the\\
    middle I don't\\
    feel\\
    so marginalized\\
\end{center}
    \end{verbatim}
    }
\end{frame}

\begin{frame}[fragile]
  \frametitle{Flushright text}
    \begin{flushright}
       This is not a political statement
    \end{flushright}
    {\color{DarkGreen}\begin{verbatim}
\begin{flushright}
    This is not a political statement
\end{flushright}
    \end{verbatim}
    }
\end{frame}

\section{Specialties}

\subsection{The index}

\begin{frame}[fragile]
  \frametitle{Indexing commands}
  \begin{description}
    \item[simple]{\color{DarkGreen}\verb+gnat\index{gnat}+}
    \item[subtopics]{\color{DarkGreen}\verb+gnat\index{gnat!size of}+}
    \item[page range]{\color{DarkGreen}\verb+\index{gnat|(}...\index{gnat|)}+}
    \item[reference]{\color{DarkGreen}\verb+\index{gnat|see{mosquito}}+}
    \item[font]{\color{DarkGreen}\verb+gnat\index{gnat@\textit{gnat}}+}
  \end{description}
  After first \LaTeX\ run, start \alert{makeindx} to sort the index.
\end{frame}

\subsection{List of acronyms}

\begin{frame}[fragile]
  \frametitle{Acronyms}
  Assume the following list of acronyms:
  {\color{DarkGreen}\begin{verbatim}
\begin{acronym}
  \acro{nfkb}[NF-\textkappa B]{nuclear factor
     \textkappa B}{protein in gene regulation}
\end{acronym}
  \end{verbatim}
  }
  This will print as\\
  NF-\textkappa B: nuclear factor \textkappa B, protein in gene regulation

  Then
  \begin{description}
    \item[{\color{DarkGreen}\(\backslash\)acf\{nfkb\}}]{nuclear factor \textkappa B (NF-\textkappa B)}
    \item[{\color{DarkGreen}\(\backslash\)acs\{nfkb\}}]{NF-\textkappa B}
    \item[{\color{DarkGreen}\(\backslash\)acl\{nfkb\}}]{nuclear factor \textkappa B}
  \end{description}
\end{frame}

\subsection{Bibliography}

\begin{frame}[fragile]
  \frametitle{BibTeX}
  Database for literature references in ASCII-format. Can be
  produced from many programs like \texttt{EndNote} or \texttt{RefMan}, also from \href{http://www.ncbi.nlm.nih.gov/pubmed/}{\texttt{\alert{PubMed}}}, \href{http://scholar.google.com/advanced_scholar_search?hl=en&as_sdt=2000}{\texttt{\alert{Google Scholar}}} (set preferences!) etc.
  {\color{DarkGreen}\small \begin{verbatim}
@article{Alb-76,
  AUTHOR= {W.J. Albery and J.R. Knowles},
  TITLE= {Evolution of enzyme function and the
    development of catalytic efficiency},
  JOURNAL= {Biochemistry},
  VOLUME= {15},
  YEAR= {1976},
  PAGES= {5631-5640},
  ABSTRACT= {Catalytic efficiency constant kcat/Km
    defined },
  DOI= {10.1021/bi00670a032},
  LANGUAGE= {engl}
}
  \end{verbatim} 
  }
  Similar for books, chapters, reports, thesis etc. In the text
  use {\color{DarkGreen}\(\backslash\)cite\{Alb-76\}}. After first \LaTeX
  -run start \alert{biber} to create the bibliography.
\end{frame}

\subsection{Slide presentations}

\begin{frame}
  \frametitle{Beamer-slides}
  {\color{DarkGreen}\(\backslash\)begin\{frame\} \\
  \hspace{2em}\(\backslash\)frametitle\{\} \\
  \hspace{2em}... \\
  \(\backslash\)end\{frame\}}
\end{frame}

\subsection{TeX-ing}

\begin{frame}
  \frametitle{\TeX ing}
  Always use the sequence:
  \begin{description}
    \item[\LaTeX]{produces the necessary intermediate
        files}
    \item[makeindx]{sort the index}
    \item[biber]{create the bibliography}
    \item[\LaTeX]{include bibliography and index, resolve
        cross-references}
    \item[\LaTeX]{resolve remaining cross-references}
  \end{description}
  Note: using pdf\LaTeX\ instead of \LaTeX\ produces pdf-files
  directly. \XeLaTeX can use any font installed for the operating system (TrueType, OpenType), supports UTF-8 and produces pdf. 
\end{frame}

\section{The departmental handout}

\begin{frame}[fragile]
  \frametitle{Packets used in dept. handout}
  \begin{tabular}{lp{0.6\textwidth}}
    acronym           & administration of acronyms                     \\
    booktabs          & professional looking tables                    \\
    chemarrow         & various types of arrows for equations          \\
    color             & colored text                                   \\
    eufrak            & font for arrays + vectors                      \\
    graphicx          & inclusion of diagrams                          \\
    hyperref          & use cross-referencing facilities of pdf        \\
    isodate           & standard conform dates                         \\
    isotope           & {\color{DarkGreen}\(\backslash\)isotope[13]{C}} = \isotope[13]{C}\\
    kpfonts           & works better than the original fonts           \\
    makeidx           & index generation                               \\
    natbib            & clean handling of bibliography                 \\
    siunitx           & {\color{DarkGreen}\(\backslash\)SI{}{}, \(\backslash\)num{}, \(\backslash\)ang{}} \\
    thumb             & chapter thumbs                                 \\
    upgreek           & {\color{DarkGreen}\(\backslash\)textalpha} = \textalpha\ instead of $\alpha$ \\
    wasysym,marvosym  & special symbols like \Male , \Female           \\
  \end{tabular}
\end{frame}

\begin{frame}
   \frametitle{The package \texttt{chemarrow}}
     \begin{description}{\small
        \item[{\color{DarkGreen}\(\backslash\)chemical\{2\(\backslash\), H\_2O\(\backslash\) \(\backslash\)chemarrow\(\backslash\) H\_3O\textasciicircum + + OH\textasciicircum -\}}]{\chemical{2\, H_2O\ \chemarrow\ H_3O^+ + OH^-}}
        \item[{\color{DarkGreen}\(\backslash\)chemical\{2\(\backslash\), H\_2O\(\backslash\) \(\backslash\)rarrowfill\{2em\}\(\backslash\) H\_3O\textasciicircum + + OH\textasciicircum -\}}]{\chemical{2\, H_2O\ \rarrowfill{2em}\ H_3O^+ + OH^-}}
        \item[{\color{DarkGreen}\(\backslash\)chemical\{2\(\backslash\), H\_2O\(\backslash\) \(\backslash\)rightleftharpoonsfill\{2em\}\(\backslash\) H\_3O\textasciicircum + + OH\textasciicircum -\}}]{\chemical{2\, H_2O\ \rightleftharpoonsfill{2em}\ H_3O^+ + OH^-}}
        \item[{\color{DarkGreen}\(\backslash\)chemical\{2\(\backslash\), H\_2O\(\backslash\) \(\backslash\)leftrightharpoonsfill\{2em\}\(\backslash\) H\_3O\textasciicircum + + OH\textasciicircum -\}}]{\chemical{2\, H_2O\ \leftrightharpoonsfill{2em}\ H_3O^+ + OH^-}}
        \item[{\color{DarkGreen}\(\backslash\)chemical\{2\(\backslash\), H\_2O\(\backslash\) \(\backslash\)autorightleftharpoons\{\$k\_a\$\}\{\$k\_b\$\}\(\backslash\) H\_3O\textasciicircum + + OH\textasciicircum -\}}]{\chemical{2\, H_2O\ \autorightleftharpoons{$k_a$}{$k_b$}\ H_3O^+ + OH^-}}
        \item[{\color{DarkGreen}\(\backslash\)chemical\{2\(\backslash\), H\_2O\(\backslash\) \(\backslash\)autoleftrightharpoons\{\$k\_a\$\}\{\$k\_b\$\}\(\backslash\) H\_3O\textasciicircum + + OH\textasciicircum -\}}]{\chemical{2\, H_2O\ \autoleftrightharpoons{$k_a$}{$k_b$}\ H_3O^+ + OH^-}}
        \item[{\color{DarkGreen}\(\backslash\)chemical\{2\(\backslash\), H\_2O\(\backslash\) \(\backslash\)autorightarrow\{\$k\_a\$\}\{\$k\_b\$\}\(\backslash\) H\_3O\textasciicircum + + OH\textasciicircum -\}}]{\chemical{2\, H_2O\ \autorightarrow{$k_a$}{$k_b$}\ H_3O^+ + OH^-}}
        \item[{\color{DarkGreen}\(\backslash\)chemical\{2\(\backslash\), H\_2O\(\backslash\) \(\backslash\)autoleftarrow\{\$\(\backslash\)triangle\$\}\{\}\(\backslash\) H\_3O\textasciicircum + + OH\textasciicircum -\}}]{\chemical{2\, H_2O\ \autoleftarrow{$\triangle$}{}\ H_3O^+ + OH^-}}
     }\end{description}
\end{frame}

\begin{frame}[fragile]
  \frametitle{Getting consistent output}
  \begin{itemize}
    \item{{\color{DarkGreen}\(\backslash\)chemical\{CH\_2\(\backslash\)double
        CH\(\backslash\)single OH\}} = \chemical{CH_2\double
        CH\single OH} }
    \item{{\color{DarkGreen}\(\backslash\)ph} = \ph , similar for
        {\color{DarkGreen}\(\backslash\)pkw, \(\backslash\)pka, \(\backslash\)poh,
        \(\backslash\)pI}}
    \item{{\color{DarkGreen}\(\backslash\)Name\{Maude Leonora Menten\}} =
        \textsc{Maude Leonora Menten}}
    \item{{\color{DarkGreen}\(\backslash\)skalar\{x\}} = $x$, similar for
        {\color{DarkGreen}\(\backslash\)array, \(\backslash\)vektor,
        \(\backslash\)set}. Note the list of variables used in
        the appendix.}
    \item{{\color{DarkGreen}\(\backslash\)SI\{1\}\{cm\}} =
        \SI{1}{cm},
        {\color{DarkGreen}\(\backslash\)num\{10000\}} =
        \num{10000},
        {\color{DarkGreen}\(\backslash\)ang\{1;2;3\}} =
        \ang{1;2;3}.}
  \end{itemize}
\end{frame}

\end{document}
